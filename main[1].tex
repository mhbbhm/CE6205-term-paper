\documentclass[12pt, letterpaper, oneside]{report}
\usepackage[lmargin=1.5in, rmargin=1in, tmargin=1in, bmargin=1in]{geometry}


\usepackage[english]{babel}
\usepackage[utf8]{inputenc}
\usepackage{lipsum}
\usepackage{mathptmx}
\usepackage{listings} 
\lstset{%
  breaklines=true
}

%\lhead{AE 747 A}
%\rhead{Molecular Gas Dynamics}
%\lfoot{Computational Assignment II}
%\rfoot{Indian Institute of Tech. Kanpur}
% some very useful LaTeX packages include:

%\usepackage{cite}      % Written by Donald Arseneau
                        % V1.6 and later of IEEEtran pre-defines the format
                        % of the cite.sty package \cite{} output to follow
                        % that of IEEE. Loading the cite package will
                        % result in citation numbers being automatically
                        % sorted and properly "ranged". i.e.,
                        % [1], [9], [2], [7], [5], [6]
                        % (without using cite.sty)
                        % will become:
                        % [1], [2], [5]--[7], [9] (using cite.sty)
                        % cite.sty's \cite will automatically add leading
                        % space, if needed. Use cite.sty's noadjust Zoption
                        % (cite.sty V3.8 and later) if you want to turn this
                        % off. cite.sty is already installed on most LaTeX
                        % systems. The latest version can be obtained at:
                        % http://www.ctan.org/tex-archive/macros/latex/contrib/supported/cite/

\usepackage{graphicx}   % Written by David Carlisle and Sebastian Rahtz
                        % Required if you want graphics, photos, etc.
                        % graphicx.sty is already installed on most LaTeX
                        % systems. The latest version and documentation can
                        % be obtained at:
                        % http://www.ctan.org/tex-archive/macros/latex/required/graphics/
                        % Another good source of documentation is "Using
                        % Imported Graphics in LaTeX2e" by Keith Reckdahl
                        % which can be found as esplatex.ps and epslatex.pdf
                        % at: http://www.ctan.org/tex-archive/info/

\usepackage{subcaption}
%\usepackage{psfrag}    % Written by Craig Barratt, Michael C. Grant,
                        % and David Carlisle
                        % This package allows you to substitute LaTeX
                        % commands for text in imported EPS graphic files.
                        % In this way, LaTeX symbols can be placed into
                        % graphics that have been generated by other
                        % applications. You must use latex->dvips->ps2pdf
                        % workflow (not direct pdf output from pdflatex) if
                        % you wish to use this capability because it works
                        % via some PostScript tricks. Alternatively, the
                        % graphics could be processed as separate files via
                        % psfrag and dvips, then converted to PDF for
                        % inclusion in the main file which uses pdflatex.
                        % Docs are in "The PSfrag System" by Michael C. Grant
                        % and David Carlisle. There is also some information
                        % about using psfrag in "Using Imported Graphics in
                        % LaTeX2e" by Keith Reckdahl which documents the
                        % graphicx package (see above). The psfrag package
                        % and documentation can be obtained at:
                        % http://www.ctan.org/tex-archive/macros/latex/contrib/supported/psfrag/

%\usepackage{subfigure} % Written by Steven Douglas Cochran
                        % This package makes it easy to put subfigures
                        % in your figures. i.e., "figure 1a and 1b"
                        % Docs are in "Using Imported Graphics in LaTeX2e"
                        % by Keith Reckdahl which also documents the graphicx
                        % package (see above). subfigure.sty is already
                        % installed on most LaTeX systems. The latest version
                        % and documentation can be obtained at:
                        % http://www.ctan.org/tex-archive/macros/latex/contrib/supported/subfigure/

\usepackage{url}        % Written by Donald Arseneau
                        % Provides better support for handling and breaking
                        % URLs. url.sty is already installed on most LaTeX
                        % systems. The latest version can be obtained at:
                        % http://www.ctan.org/tex-archive/macros/latex/contrib/other/misc/
                        % Read the url.sty source comments for usage information.

%\usepackage{stfloats}  % Written by Sigitas Tolusis
                        % Gives LaTeX2e the ability to do double column
                        % floats at the bottom of the page as well as the top.
                        % (e.g., "\begin{figure*}[!b]" is not normally
                        % possible in LaTeX2e). This is an invasive package
                        % which rewrites many portions of the LaTeX2e output
                        % routines. It may not work with other packages that
                        % modify the LaTeX2e output routine and/or with other
                        % versions of LaTeX. The latest version and
                        % documentation can be obtained at:
                        % http://www.ctan.org/tex-archive/macros/latex/contrib/supported/sttools/
                        % Documentation is contained in the stfloats.sty
                        % comments as well as in the presfull.pdf file.
                        % Do not use the stfloats baselinefloat ability as
                        % IEEE does not allow \baselineskip to stretch.
                        % Authors submitting work to the IEEE should note
                        % that IEEE rarely uses double column equations and
                        % that authors should try to avoid such use.
                        % Do not be tempted to use the cuted.sty or
                        % midfloat.sty package (by the same author) as IEEE
                        % does not format its papers in such ways.

\usepackage{amsmath}    % From the American Mathematical Society
                        % A popular package that provides many helpful commands
                        % for dealing with mathematics. Note that the AMSmath
                        % package sets \interdisplaylinepenalty to 10000 thus
                        % preventing page breaks from occurring within multiline
                        % equations. Use:
%\interdisplaylinepenalty=2500
                        % after loading amsmath to restore such page breaks
                        % as IEEEtran.cls normally does. amsmath.sty is already
                        % installed on most LaTeX systems. The latest version
                        % and documentation can be obtained at:
                        % http://www.ctan.org/tex-archive/macros/latex/required/amslatex/math/

\usepackage{float}

% Other popular packages for formatting tables and equations include:

%\usepackage{array}
% Frank Mittelbach's and David Carlisle's array.sty which improves the
% LaTeX2e array and tabular environments to provide better appearances and
% additional user controls. array.sty is already installed on most systems.
% The latest version and documentation can be obtained at:
% http://www.ctan.org/tex-archive/macros/latex/required/tools/

% V1.6 of IEEEtran contains the IEEEeqnarray family of commands that can
% be used to generate multiline equations as well as matrices, tables, etc.

% Also of notable interest:
% Scott Pakin's eqparbox package for creating (automatically sized) equal
% width boxes. Available:
% http://www.ctan.org/tex-archive/macros/latex/contrib/supported/eqparbox/

% *** Do not adjust lengths that control margins, column widths, etc. ***
% *** Do not use packages that alter fonts (such as pslatex).         ***
% There should be no need to do such things with IEEEtran.cls V1.6 and later.

\usepackage[toc,page]{appendix}
% Your document starts here!
\begin{document}

\begin{titlepage}

\newcommand{\HRule}{\rule{\linewidth}{0.5mm}} % Defines a new command for the horizontal lines, change thickness here

\center % Center everything on the page
 
%----------------------------------------------------------------------------------------
%	HEADING SECTIONS
%----------------------------------------------------------------------------------------

\textsc{\Huge \bf Bangladesh University of Engineering and Technology}\\[0.5cm] % Minor heading such as course title
\textsc{\Large Department of Civil Engineering}\\[0.1cm]
%----------------------------------------------------------------------------------------
%	TITLE SECTION
%----------------------------------------------------------------------------------------

\HRule \\[0.4cm]
{ \huge \bfseries Term Project}\\[0.4cm] % Title of your document
{ \Large \bfseries Evaluation of the temperature of fire-exposed columns using a simplified method and finite element approach}\\[0.2cm]
\HRule \\[1.5cm]
 
%----------------------------------------------------------------------------------------
%	AUTHOR SECTION
%----------------------------------------------------------------------------------------


\begin{minipage}{0.8\textwidth}
\begin{center} \large
Course Teacher: \\Dr. Ishtiaque Ahmed \\and \\Dr. Tanvir Manzur
\end{center}

\begin{center} \large
Submitted by: \\Md. Mehedi Hassan Bhuiyan \\0421042311
\end{center}

\begin{center}
November 15, 2022
\end{center}

\end{minipage}


\begin{figure}[h]
\centerline{\includegraphics[scale=0.35]{Buet-Logo.png}}

\label{logo}
\end{figure}



\end{titlepage}

\pagenumbering{roman}
\begin{center}
 \addcontentsline{toc}{chapter}{Abstract}
\section*{Abstract}
\end{center}
This term project compares two simplified models for figuring out the temperature of reinforcements with a tested finite element method. A total of 112 models are created in this study using two different cross-sectional columns; each column has two different effective cover (50 mm and 75 mm) specimens that are subjected to different durations (30, 60, 90, 120, 150, 180, and 240 minutes) of standard fire tests in a finite element environment using a verified method. According to the findings of this study, simplified methods and FE outputs had an average 20\% error in temperature evaluation for 60- to 240-min fire-exposed models. Furthermore, when those columns are exposed to fire on two, three, or all of their faces, the temperature in the reinforcement calculated using the simplified method has a margin of error of less than 15\% for the 90–240 minute range duration. But when column samples are exposed to a 30-minute fire, both simplified methods show a wider range of percentage errors. There was no difference in temperature between the simplified and FEA ways of reinforcing square or rectangular columns. This is because the location of the reinforcement was the same in both types of columns because the same effective cover was used. Both simplified methods can be used for reinforcement temperature evaluation purposes for structures exposed to fire for more than 60 minutes, but for better output, proposed verified FEA approaches may be adapted.
\clearpage
\pagebreak
\tableofcontents
\listoffigures
\thispagestyle{empty}
\pagebreak

\cleardoublepage
\clearpage
\pagenumbering{arabic}
\setcounter{page}{1}
\renewcommand\thesection{\arabic{section}}

\section{Introduction}

A fire hazard is an unwanted event in this world. Any structure can be affected by this event, but specially reinforced concrete structures may be less affected than other materials because the material properties of the outer core concrete degrade slower than the inner steel reinforcement \cite{R6}. A few research groups have come up with a simplified way to figure out the residual capacity of RC beams \cite{R1,R5} and RC columns \cite{R4} after a fire event, and both of these simplified methods need to know the temperature of the reinforcements. It is hard to get a more precise look at the temperature of the inner core reinforcement. Most of the time, engineers and their concerns depend on firefighter reports to find out the temperature of this structure. In reality, fire fighters need some time to reach the fire event place, and most of the time it is not possible to delineate the real event scenario in a report. Assuming that firefighters report actual fire event durations and peak temperatures on their reports, it is time-consuming and costly to determine the actual fire temperature of inner reinforcement by advance machinery or finite element approaches. To overcome this problem, there are many simplified methods available to determine the inner core reinforcement temperature that are easy, less time-consuming, and less costly. Among them, we discuss two simplified methods that are proposed by Wickstrom \cite{R3} and Kodur \cite{R2}.\\
\\
Wickstrom \cite{R3} introduces two methods for determining the temperature of structures: one for one-dimensional heat transfer and another for two-dimensional heat transfer. The author proposes equations (\ref{Eq1})-(\ref{Eq3}) to evaluate temperature for 1D heat transfer purposes, and for 2D heat transfer purposes, equations (\ref{Eq1}), (\ref{Eq3}) and (\ref{Eq4}) are used. In equations (\ref{Eq1})-(\ref{Eq3}) \(\eta_w\) is the ratio of the temperature at the surface of the concrete to the temperature of the fire, \(\eta_z\) is the heat transfer factor induced through a single surface that is exposed to the fire, and \(T_f\) is the temperature of the fire. The heat conduction that takes place in two different directions (z and y) needs to be accounted for in order to obtain temperatures at corner locations of a structural member. This is done through the use of \(\eta_z\) and \(\eta_y\), with \(\eta_y\) being calculated in the same manner as \(\eta_z\) in the equation. (\ref{Eq2}). The temperature Tc that is reached after being exposed to fire in both the x and y directions is as follows. Where \(\eta_z\) and \(\eta_y\) represent the heat transfer factors that are caused by exposure on either side of the fire. Due to the fact that it does not take into account the various rates at which temperatures can increase during a fire \cite{R7}, Wickstrom's empirical equation can only provide a rough estimate of the cross-sectional temperatures. Additionally, the influence of aggregate type (siliceous or carbonate), newer concrete types (high strength concrete), and the variation of thermal properties with temperature are not accounted for in this equation. As a result, it is possible that this equation is not applicable to various types of concrete.

\begin{equation}\label{Eq1}
    T_c=\eta_z\ \eta_w\ T_f
\end{equation}
\begin{equation}\label{Eq2}
    \eta_z=0.18ln\left(\frac{t_h}{z^2}\right)-0.81
\end{equation}
\begin{equation}\label{Eq3}
    \eta_w=1-0.0616t_h^{-0.88}
\end{equation}
\begin{equation}\label{Eq4}
    T_c=\left[\eta_w\left(\eta_z+\eta_y-{2\eta}_z\eta_y\right)+\eta_z\eta_y\right]T_f
\end{equation}
\\
By taking aggregate type into account, Kodur \cite{R2} overcomes the gap in Wickstrom's proposed method. Kodur approaches consider aggregate types like carbonate aggregate (CA) and silicious aggregate (SA), as well as types of concrete such as normal-strength concrete (NSC) and high-strength concrete (HSC). Following Wickstrom, Kodur also provides two methods for considering 1D and 2D heat transfer. Kodur proposes equations (\ref{Eq5}) and (\ref{Eq6}) to evaluate temperature for 1D heat transfer purposes, and for 2D heat transfer purposes, equations (\ref{Eq6}) and (\ref{Eq7}) are used.

\begin{equation}\label{Eq5}
    T_c=c_1\eta_z\left(at^n\right)
\end{equation}
\begin{equation}\label{Eq6}
    \eta_z=0.155ln\left(\frac{t}{z^{1.5}}\right)-0.348\sqrt z-0.371
\end{equation}
\begin{equation}\label{Eq7}
    T_c=c_2\left[-1.481\left(\eta_z\eta_y\right)+0.985\left(\eta_z+\eta_y\right)+0.017\right]\left(at^n\right)
\end{equation}
\\
For ISO 834 \cite{R8} fire, a=935 and n=0.168, and for ASTM E119 \cite{R9} fire, a=910 and n=0.148 and \(c_1\) are 1.0, 1.01, 1.12 and 1.12 for NSC-CA, HSC-CA, NSC-SA and HSC-SA, respectively; \(c_2\) are 1.0, 1.06, 1.12 and 1.20 for NSC-CA, HSC-CA, NSC-SA and HSC-SA, respectively. The 1D and 2D heat transfer and coordinate systems are depicted in the y and z directions, respectively, in Figure \ref{fig:1}.
\\
\\
The two simplified methods mentioned above are widely used in the temperature calculation of the Reinforce structure element. As mentioned before, rebar temperature plays an important role in estimating residual capacity. So, the goal of this term's project is to compare how accurately these two methods measure the temperature of the rebar to the verified FE computational method.
\begin{figure}
    \centering
    \includegraphics[scale=0.55]{Heat.PNG}
    \caption{Separation of concrete member into areas for use in determining internal temperatures.}
    \label{fig:1}
\end{figure}

\section{Thermal Analysis}
A computational FE approach is required to evaluate the temperature of the rebar in a fire-exposed column. To do computational thermal analysis and create an environment similar to a fire event, the model needs to include properties of concrete and rebar that change with temperature. Material properties and FE modeling approaches are described in detail in the sections that follow.

\subsection{Thermal Property}
Eurocode 2 \cite{R10} properties are widely used for FEA verification purposes. Kodur \cite{R4, R5, R11}, Bhuiyan \cite{R1}, and other research groups also use Eurocode 2 properties for verification in FE analysis. For thermal analysis of concrete and rebar density, specific heat and conductivity are required, which are followed as per Eurocode 2. As per the code specified, normal-strength concrete density is 2300 \(kg/m^3\), and steel reinforcement density is 7850 \(kg/m^3\) as used in the FE model. Also, temperature-dependent specific heat and conductivity are implemented in this model as well. The thermal conductivity of both concrete and steel is presented in Figure \ref{fig:2}. The code specifies that any reasonable concrete thermal conductivity may be implemented in between the upper and lower bound ranges in the FE model. A research group verified their experimental result with the FE model and found a better result when lower bound thermal conductivity was implemented as a material property of the concrete element. Also, for a fire-exposed specimen with an exposed surface, the convective heat transfer coefficient is taken to be 25 \(W/m^2k\) \cite{R10} and \(W/m^2k\) for an unexposed surface \cite{R13}. The emissivity for radiative heat transfer at the exposed surfaces of the concrete member is taken as 0.8, Stefan-Boltzmann constant \(5.67\times{10}^{-8}\ W/m^{2}k^{4}\) and absolute zero temperature was taken -273.16 \(K\). As mentioned, properties were also incorporated into this study for better verification purposes.
\begin{figure}
    \centering
    \includegraphics[scale=0.8]{G1.PNG}
    \caption{Thermal Property Conductivity as per EC-2. \cite{R10}}
    \label{fig:2}
\end{figure}
\subsection{FE Modeling Approach}
To make sure the concrete and rebar are connected, a few segments are made and tied together. This helps the heat move from the concrete to the rebar. For heat transfer modeling purposes, the FE package Abaqus was used in this study, with a concrete element using DC3D8 and a rebar element using DC1D2. Finally, the concrete and rebar are applied to a 25-mm mesh. To evaluate the concrete and rebar temperatures, a few sets of nodes were created in this model, which work as a thermocouple to read the specific location temperature over time. Figure \ref{fig:3} depicts a post-analysis contour diagram, which also shows element type and exposed and unexposed surface conditions.
\begin{figure}
    \centering
    \includegraphics[scale=0.5]{G2.png}
    \caption{Fire-exposed finite element model}
    \label{fig:3}
\end{figure}

\section{FE Model Validation}
An experimental study \cite{R11} was recreated in the FE environment, and all dependency was incorporated as per the discussion in Section 2. There is also a furnace temperature implemented in this model, and a predefined field impliment that the specimen is at room temperature. All four faces of the column in the central 1.7-meter height were exposed to fire, as shown in Figure \ref{fig:4}. The column was tested for a 90-minute heating phase that followed that of ASTM E119 \cite{R9}. The experimental and FEA results are plotted and shown in Figure \ref{fig:5}. From this plot, it is clearly seen that the experimental and FEA rebar temperatures are almost similar, but the concrete center temperature shows some deficit in the last 15 minutes of fire duration. After taking into account all of the factors in the FE model, the output results are also in line with the experimental results. This means that the approach has been verified and is ready for more parametric study.
\begin{figure}
    \centering
    \includegraphics[scale=0.65]{G10.png}
    \caption{Dimensions and reinforcement details of RC columns.}
    \label{fig:4}
\end{figure}
\begin{figure}
    \centering
    \includegraphics[scale=0.9]{G4.PNG}
    \caption{Comparison of predicted (FEA) and measured column temperatures during fire exposure. \cite{R11}}
    \label{fig:5}
\end{figure}
\section{Parametric Study}
After the experimental results were validated in a finite element environment, similar methods are now being used for the parametric study. A number of studies are conducted in an FE environment to determine which simplified method provides better precision rebar temperature. Two different cross-sectional columns were chosen for this study, and each specimen was created for two effective cover sizes, which are 50 and 75 mm. These four categories are now being tested under standard fire \cite{R9} exposure for 30, 60, 90, 120, 150, 180, and 240 minute durations. Also, each specimen is exposed to fire on one face, two faces, three faces, and all faces of columns. Now, the number of studies required (2 column size x 2 effective covers x 7 fire exposure durations x 4 exposure faces = 112 different models) in the finite element model is 112. The rectangular column specimens cross-sectional profiles and fire exposure faces are described in Figure \ref{fig:6}, similar follows for square columns.\\
Preparing 112 models manually in a finite element environment is a time-consuming process. To reduce model preparation time, the Abaqus scripting \cite{R14} approach is adopted. This scripting approach helps create all of those models, and using this approach, anyone can prepare any number of models within a short period of time. The detailed scripting approaches are mentioned in Appendix A.
\\
\begin{figure}
    \centering
    \includegraphics[scale=0.5]{G9.PNG}
    \caption{Selected specimens for study (a) one exposed side (b) two exposed sides (c) three exposed sides (d) four exposed sides to ASTM E119 \cite{R9} standard fire.}
    \label{fig:6}
\end{figure}
\\
Because the effective cover-hold rebar position was the same for rectangular and square-shaped columns, extensive computational studies revealed that the rebar temperature remained constant. For this reason, the results of square column 56 studies are not shown in this term paper to avoid repetition. So, the remaining 56 studies are presented below.\\
\\
Before talking about the results of parametric studies, it's important to explain how specimen ID is written. Such as "F240-C50" for a specimen ID, where "F" stands for "fire exposure," "240" stands for "240 minutes" of exposure, and "C50" stands for "50 mm effective cover." At first, discuss the finite element analysis results of one-sided fire exposure in rectangular columns. Compare the FE output among Wickstrom and Kodur simplified methods, as displayed in Figure \ref{fig:7}. According to the studies, the simplified method performs poorly for 30-minute fire durations and performs better for 60-240 minute fire durations. There was no significant error found in the rebar temperature estimation by the simplified method with respect to varying the effective cover size of columns.\\
\\
Figures \ref{fig:8}, \ref{fig:9}, and \ref{fig:10} show that there is a similar temperature found in "B4" rebar for fire exposure in two-, three-, and four-faced columns because "B4" rebar experiences 2D heat transfer in both cases. Wickstrom's proposed methods show higher precision in the estimation of rebar temperature in comparison to the Kodur method during 60–240 minute fire durations. But for a 30-minute fire duration, similar abrupt behavior was found in both methods. There is a significant effect noticed in Kodur's proposed method for 75 mm effective cover that shows more precision in estimating rebar temperature than a 50 mm effective cover column specimen during 150–240 minute fire durations.
\begin{figure}
    \centering
    \includegraphics[scale=0.5]{G5.PNG}
    \caption{The temperature of Rebar exposed to fire on 1 side for different durations}
    \label{fig:7}
\end{figure}
\begin{figure}
    \centering
    \includegraphics[scale=0.5]{G6.PNG}
    \caption{The temperature of Rebar exposed to fire on 2 side for different durations}
    \label{fig:8}
\end{figure}
\begin{figure}
    \centering
    \includegraphics[scale=0.5]{G7.PNG}
    \caption{The temperature of Rebar exposed to fire on 3 side for different durations}
    \label{fig:9}
\end{figure}
\begin{figure}
    \centering
    \includegraphics[scale=0.7]{G8.PNG}
    \caption{The temperature of Rebar exposed to fire on 3 side for different durations}
    \label{fig:10}
\end{figure}

\section{Conclusion}
Following an extensive 112 FE computational model study and comparison with two simplified temperature estimation techniques proposed by Wickstrom and Kodur, a few conclusions were reached. The following are the ideas generated or conclusions drawn from this study.
\begin{enumerate}
  \item For Rebar B4, 30 min fire (C75) exposure shows an unusual   error found in both simplified methods in all conditions (1, 2, 3, & 4 side fire exposure).
  \item The C50 Rebar B4 for one-sided fire exposure specimen shows less than a 15\% error in temperature evaluation in the
Wickstrom method and less than a 24\% error in the Kodur proposed method. And the C75 Rebar B4 specimen for one sided fire exposure shows less than a 20\% error in both methods.
  \item Both methods show less than 15\% error in temperature evaluation for C50 and C75 specimens' fire exposure on 2, 3,
and 4 sides with 90-240 min fire exposure.
  \item The percentage error in temperature estimation decreases as fire duration decreases (240-60 min range) in 1 side fire
exposure in the Kodur approach. But no increasing or decreasing pattern was found for 2, 3, and 4 side fire exposure
in both methods.
  \item For a longer duration of fire exposure (60-240 min range), both simplified methods show an average 20% error in
temperature evaluation. Both simplified methods can be used in the temperature evaluation process for the above mentioned duration, but FEA can be used for greater precision.
\end{enumerate}
 
\newpage
\bibliographystyle{sn-standardnature}
\addcontentsline{toc}{chapter}{References}
\renewcommand{\bibname}{References}
\bibliography{Ref}

\newpage
\appendix
\chapter{Abaqus Scripting with python}
This is my first Appendix .

\lstset{numbers=left, language=python}
\begin{lstlisting}[frame=none]
from part import *
from material import *
from section import *
from assembly import *
from step import *
from interaction import *
from load import *
from mesh import *
from optimization import *
from job import *
from sketch import *
from visualization import *
from connectorBehavior import *

session.journalOptions.setValues(replayGeometry=COORDINATE, recoverGeometry=COORDINATE)

#Create Part
W=500
D=500
H=4000
C=75
C1=0.5*W-C
C2=0.5*D-C
t=240 #input as min
d=25  #Mesh Size
b=H/(2*d)

mdb.models['Model-1'].setValues(absoluteZero=-273.16, stefanBoltzmann=5.667e-11)
mdb.models['Model-1'].ConstrainedSketch(name='__profile__', sheetSize=200.0)
mdb.models['Model-1'].sketches['__profile__'].rectangle(point1=(0.5*W, 0.5*D), point2=(-0.5*W, -0.5*D))
mdb.models['Model-1'].Part(dimensionality=THREE_D, name='Beam', type=DEFORMABLE_BODY)
mdb.models['Model-1'].parts['Beam'].BaseSolidExtrude(depth=H, sketch=mdb.models['Model-1'].sketches['__profile__'])

mdb.models['Model-1'].ConstrainedSketch(name='__profile__', sheetSize=200.0)
mdb.models['Model-1'].sketches['__profile__'].Line(point1=(0.0, 0.0), point2=(H, 0.0))
mdb.models['Model-1'].sketches['__profile__'].HorizontalConstraint(addUndoState=False, entity=mdb.models['Model-1'].sketches['__profile__'].geometry[2])
mdb.models['Model-1'].Part(dimensionality=THREE_D, name='Bar', type=DEFORMABLE_BODY)
mdb.models['Model-1'].parts['Bar'].BaseWire(sketch=mdb.models['Model-1'].sketches['__profile__'])

#Cell

mdb.models['Model-1'].parts['Beam'].DatumPlaneByPrincipalPlane(offset=0.0, principalPlane=YZPLANE)
mdb.models['Model-1'].parts['Beam'].DatumPlaneByPrincipalPlane(offset=0.0, principalPlane=XZPLANE)
mdb.models['Model-1'].parts['Beam'].DatumPlaneByPrincipalPlane(offset=C1, principalPlane=YZPLANE)
mdb.models['Model-1'].parts['Beam'].DatumPlaneByPrincipalPlane(offset=-C1, principalPlane=YZPLANE)
mdb.models['Model-1'].parts['Beam'].DatumPlaneByPrincipalPlane(offset=C2, principalPlane=XZPLANE)
mdb.models['Model-1'].parts['Beam'].DatumPlaneByPrincipalPlane(offset=-C2, principalPlane=XZPLANE)

#Cut

mdb.models['Model-1'].parts['Beam'].PartitionCellByDatumPlane(cells=mdb.models['Model-1'].parts['Beam'].cells.findAt(((-0.5*W, -0.5*D, H), )), datumPlane=mdb.models['Model-1'].parts['Beam'].datums[2])
mdb.models['Model-1'].parts['Beam'].PartitionCellByDatumPlane(cells=mdb.models['Model-1'].parts['Beam'].cells.findAt(((0.5*W, 0.5*D, H), ),((-0.5*W, -0.5*D, H),),), datumPlane=mdb.models['Model-1'].parts['Beam'].datums[3])
mdb.models['Model-1'].parts['Beam'].PartitionCellByDatumPlane(cells=mdb.models['Model-1'].parts['Beam'].cells.findAt(((0.5*W, 0.5*D, H), ),((0.5*W, -0.5*D, H),),), datumPlane=mdb.models['Model-1'].parts['Beam'].datums[4])
mdb.models['Model-1'].parts['Beam'].PartitionCellByDatumPlane(cells=mdb.models['Model-1'].parts['Beam'].cells.findAt(((-0.5*W, 0.5*D, H), ),((-0.5*W, -0.5*D, H),),), datumPlane=mdb.models['Model-1'].parts['Beam'].datums[5])
mdb.models['Model-1'].parts['Beam'].PartitionCellByDatumPlane(cells=mdb.models['Model-1'].parts['Beam'].cells.findAt(((0.5*W, 0.5*D, H), ),((-0.5*W, 0.5*D, H),),((-0.5*C1, 0.5*D, H),),((0.5*C1, 0.5*D, H),),), datumPlane=mdb.models['Model-1'].parts['Beam'].datums[6])
mdb.models['Model-1'].parts['Beam'].PartitionCellByDatumPlane(cells=mdb.models['Model-1'].parts['Beam'].cells.findAt(((0.5*W, -0.5*D, H), ),((-0.5*W, -0.5*D, H),),((-0.5*C1, -0.5*D, H),),((0.5*C1, -0.5*D, H),),), datumPlane=mdb.models['Model-1'].parts['Beam'].datums[7])

#Material

mdb.models['Model-1'].Material(name='Concrete')
mdb.models['Model-1'].materials['Concrete'].Density(table=((2.3e-09, ), ))
mdb.models['Model-1'].materials['Concrete'].Conductivity(table=((1.33, 20.0), (
    1.23, 100.0), (1.11, 200.0), (1.0, 300.0), (0.91, 400.0), (0.82, 500.0), (
    0.75, 600.0), (0.69, 700.0), (0.64, 800.0), (0.6, 900.0), (0.57, 1000.0), (
    0.55, 1100.0), (0.55, 1200.0)), temperatureDependency=ON)
mdb.models['Model-1'].materials['Concrete'].SpecificHeat(table=((900000000.0, 
    20.0), (900000000.0, 100.0), (1000000000.0, 200.0), (1050000000.0, 300.0), 
    (1100000000.0, 400.0), (1100000000.0, 500.0), (1100000000.0, 600.0), (
    1100000000.0, 700.0), (1100000000.0, 800.0), (1100000000.0, 900.0), (
    1100000000.0, 1000.0), (1100000000.0, 1100.0), (1100000000.0, 1200.0)), 
    temperatureDependency=ON)

mdb.models['Model-1'].Material(name='Steel')
mdb.models['Model-1'].materials['Steel'].Density(table=((7.85e-09, ), ))
mdb.models['Model-1'].materials['Steel'].Conductivity(table=((53.3, 20.0), (
    50.7, 100.0), (47.3, 200.0), (44.0, 300.0), (40.7, 400.0), (37.4, 500.0), (
    34.0, 600.0), (30.7, 700.0), (27.3, 800.0), (27.3, 900.0), (27.3, 1000.0), 
    (27.3, 1100.0), (27.3, 1200.0)), temperatureDependency=ON)
mdb.models['Model-1'].materials['Steel'].SpecificHeat(table=((439801760.0, 
    20.0), (487620000.0, 100.0), (529760000.0, 200.0), (564740000.0, 300.0), (
    605880000.0, 400.0), (666500000.0, 500.0), (760217391.3, 600.0), (
    1008157895.0, 700.0), (5000000000.0, 735.0), (803260869.6, 800.0), (
    650000000.0, 900.0), (650000000.0, 1000.0), (650000000.0, 1100.0), (
    650000000.0, 1200.0)), temperatureDependency=ON)

mdb.models['Model-1'].HomogeneousSolidSection(material='Concrete', name='C', thickness=None)
mdb.models['Model-1'].TrussSection(area=200.0, material='Steel', name='S')
mdb.models['Model-1'].parts['Beam'].SectionAssignment(offset=0.0, offsetField='', offsetType=MIDDLE_SURFACE, region=Region(cells=mdb.models['Model-1'].parts['Beam'].cells.findAt(((0.5*C1, 0.5*C2, H), ), ((-0.5*C1, 0.5*C2, H), ), ((0.5*C1, -0.5*C2, H), ), ((-0.5*C1, -0.5*C2, H), ), ((0.5*C1, 0.5*D, H), ), ((0.5*C1, -0.5*D, H), ), ((-0.5*C1, 0.5*D, H), ), ((-0.5*C1, -0.5*D, H), ), ((1.05*C1, 0.5*C2, H), ), ((1.05*C1, -0.5*C2, H), ), ((-1.05*C1, 0.5*C2, H), ), ((-1.05*C1, -0.5*C2, H), ), ((0.45*W, 0.45*D, H), ), ((0.45*W, -0.45*D, H), ), ((-0.45*W, 0.45*D, H), ), ((-0.45*W, -0.45*D, H), ), )), sectionName='C', thicknessAssignment=FROM_SECTION)
mdb.models['Model-1'].parts['Bar'].SectionAssignment(offset=0.0, offsetField='', offsetType=MIDDLE_SURFACE, region=Region(edges=mdb.models['Model-1'].parts['Bar'].edges.findAt(((0.5*H, 0.0, 0.0), ), )), sectionName='S', thicknessAssignment=FROM_SECTION)

#Assembly

mdb.models['Model-1'].rootAssembly.DatumCsysByDefault(CARTESIAN)
mdb.models['Model-1'].rootAssembly.Instance(dependent=ON, name='Bar-1', part=mdb.models['Model-1'].parts['Bar'])
mdb.models['Model-1'].rootAssembly.LinearInstancePattern(direction1=(1.0, 0.0, 0.0), direction2=(0.0, 1.0, 0.0), instanceList=('Bar-1', ), number1=1, number2=2, spacing1=H, spacing2=2*C2)
mdb.models['Model-1'].rootAssembly.rotate(angle=-90.0, axisDirection=(0.0, 2*C2, 0.0), axisPoint=(0.0, 0.0, 0.0), instanceList=('Bar-1', 'Bar-1-lin-1-2'))
mdb.models['Model-1'].rootAssembly.LinearInstancePattern(direction1=(1.0, 0.0, 0.0), direction2=(0.0, 1.0, 0.0), instanceList=('Bar-1', 'Bar-1-lin-1-2'), number1=2, number2=1, spacing1=2*C1, spacing2=2*C2)
mdb.models['Model-1'].rootAssembly.Instance(dependent=ON, name='Beam-1', part=mdb.models['Model-1'].parts['Beam'])
mdb.models['Model-1'].rootAssembly.translate(instanceList=('Bar-1', 'Bar-1-lin-1-2', 'Bar-1-lin-2-1', 'Bar-1-lin-1-2-lin-2-1'), vector=(-C1, -C2, 0.0))


#Step

mdb.models['Model-1'].HeatTransferStep(deltmx=25.0, initialInc=10.0, maxInc=200.0, maxNumInc=10000000, minInc=0.1, name='Fire', previous='Initial', timePeriod=t*60)
mdb.models['Model-1'].fieldOutputRequests['F-Output-1'].setValues(variables=('NT', ))

#Amplitude

mdb.models['Model-1'].TabularAmplitude(data=((0.0, 20.0), (60.0, 332.5304157), 
    (120.0, 426.044443), (180.0, 487.12866), (240.0, 532.516808), (300.0, 
    568.4577615), (360.0, 598.0496271), (420.0, 623.0742538), (480.0, 
    644.6569038), (540.0, 663.5558175), (600.0, 680.3069656), (660.0, 
    695.303428), (720.0, 708.8420627), (780.0, 721.1524826), (840.0, 
    732.4158538), (900.0, 742.7775408), (960.0, 752.3558747), (1020.0, 
    761.248392), (1080.0, 769.5363737), (1140.0, 777.2882137), (1200.0, 
    784.5619596), (1260.0, 791.4072598), (1320.0, 797.8668765), (1380.0, 
    803.9778734), (1440.0, 809.7725613), (1500.0, 815.2792562), (1560.0, 
    820.5228938), (1620.0, 825.5255316), (1680.0, 830.3067617), (1740.0, 
    834.8840538), (1800.0, 839.2730404)), name='Fire-30', smooth=SOLVER_DEFAULT
    , timeSpan=STEP)
	
mdb.models['Model-1'].TabularAmplitude(data=((0.0, 20.0), (60.0, 332.5304157), 
    (120.0, 426.044443), (180.0, 487.12866), (240.0, 532.516808), (300.0, 
    568.4577615), (360.0, 598.0496271), (420.0, 623.0742538), (480.0, 
    644.6569038), (540.0, 663.5558175), (600.0, 680.3069656), (660.0, 
    695.303428), (720.0, 708.8420627), (780.0, 721.1524826), (840.0, 
    732.4158538), (900.0, 742.7775408), (960.0, 752.3558747), (1020.0, 
    761.248392), (1080.0, 769.5363737), (1140.0, 777.2882137), (1200.0, 
    784.5619596), (1260.0, 791.4072598), (1320.0, 797.8668765), (1380.0, 
    803.9778734), (1440.0, 809.7725613), (1500.0, 815.2792562), (1560.0, 
    820.5228938), (1620.0, 825.5255316), (1680.0, 830.3067617), (1740.0, 
    834.8840538), (1800.0, 839.2730404), (1860.0, 843.4877575), (1920.0, 
    847.5408468), (1980.0, 851.4437275), (2040.0, 855.2067436), (2100.0, 
    858.8392889), (2160.0, 862.349916), (2220.0, 865.746429), (2280.0, 
    869.0359653), (2340.0, 872.2250658), (2400.0, 875.3197372), (2460.0, 
    878.3255053), (2520.0, 881.2474635), (2580.0, 884.0903142), (2640.0, 
    886.8584059), (2700.0, 889.5557668), (2760.0, 892.1861333), (2820.0, 
    894.7529764), (2880.0, 897.2595254), (2940.0, 899.708788), (3000.0, 
    902.1035698), (3060.0, 904.4464908), (3120.0, 906.7400006), (3180.0, 
    908.9863921), (3240.0, 911.187814), (3300.0, 913.3462819), (3360.0, 
    915.4636885), (3420.0, 917.5418131), (3480.0, 919.5823298), (3540.0, 
    921.5868151), (3600.0, 923.5567552)), name='Fire-60', smooth=SOLVER_DEFAULT
    , timeSpan=STEP)
	
mdb.models['Model-1'].TabularAmplitude(data=((0.0, 20.0), (60.0, 332.5304157), 
    (120.0, 426.044443), (180.0, 487.12866), (240.0, 532.516808), (300.0, 
    568.4577615), (360.0, 598.0496271), (420.0, 623.0742538), (480.0, 
    644.6569038), (540.0, 663.5558175), (600.0, 680.3069656), (660.0, 
    695.303428), (720.0, 708.8420627), (780.0, 721.1524826), (840.0, 
    732.4158538), (900.0, 742.7775408), (960.0, 752.3558747), (1020.0, 
    761.248392), (1080.0, 769.5363737), (1140.0, 777.2882137), (1200.0, 
    784.5619596), (1260.0, 791.4072598), (1320.0, 797.8668765), (1380.0, 
    803.9778734), (1440.0, 809.7725613), (1500.0, 815.2792562), (1560.0, 
    820.5228938), (1620.0, 825.5255316), (1680.0, 830.3067617), (1740.0, 
    834.8840538), (1800.0, 839.2730404), (1860.0, 843.4877575), (1920.0, 
    847.5408468), (1980.0, 851.4437275), (2040.0, 855.2067436), (2100.0, 
    858.8392889), (2160.0, 862.349916), (2220.0, 865.746429), (2280.0, 
    869.0359653), (2340.0, 872.2250658), (2400.0, 875.3197372), (2460.0, 
    878.3255053), (2520.0, 881.2474635), (2580.0, 884.0903142), (2640.0, 
    886.8584059), (2700.0, 889.5557668), (2760.0, 892.1861333), (2820.0, 
    894.7529764), (2880.0, 897.2595254), (2940.0, 899.708788), (3000.0, 
    902.1035698), (3060.0, 904.4464908), (3120.0, 906.7400006), (3180.0, 
    908.9863921), (3240.0, 911.187814), (3300.0, 913.3462819), (3360.0, 
    915.4636885), (3420.0, 917.5418131), (3480.0, 919.5823298), (3540.0, 
    921.5868151), (3600.0, 923.5567552), (3660.0, 925.4935524), (3720.0, 
    927.3985305), (3780.0, 929.272941), (3840.0, 931.1179673), (3900.0, 
    932.9347297), (3960.0, 934.7242894), (4020.0, 936.4876525), (4080.0, 
    938.2257733), (4140.0, 939.9395579), (4200.0, 941.629867), (4260.0, 
    943.2975188), (4320.0, 944.9432918), (4380.0, 946.5679266), (4440.0, 
    948.1721291), (4500.0, 949.7565716), (4560.0, 951.3218955), (4620.0, 
    952.8687125), (4680.0, 954.3976068), (4740.0, 955.9091364), (4800.0, 
    957.4038344), (4860.0, 958.8822108), (4920.0, 960.3447534), (4980.0, 
    961.7919291), (5040.0, 963.224185), (5100.0, 964.6419498), (5160.0, 
    966.0456339), (5220.0, 967.4356315), (5280.0, 968.8123203), (5340.0, 
    970.1760633), (5400.0, 971.5272087)), name='Fire-90', smooth=SOLVER_DEFAULT
    , timeSpan=STEP)
	
mdb.models['Model-1'].TabularAmplitude(data=((0.0, 20.0), (60.0, 332.5304157), 
    (120.0, 426.044443), (180.0, 487.12866), (240.0, 532.516808), (300.0, 
    568.4577615), (360.0, 598.0496271), (420.0, 623.0742538), (480.0, 
    644.6569038), (540.0, 663.5558175), (600.0, 680.3069656), (660.0, 
    695.303428), (720.0, 708.8420627), (780.0, 721.1524826), (840.0, 
    732.4158538), (900.0, 742.7775408), (960.0, 752.3558747), (1020.0, 
    761.248392), (1080.0, 769.5363737), (1140.0, 777.2882137), (1200.0, 
    784.5619596), (1260.0, 791.4072598), (1320.0, 797.8668765), (1380.0, 
    803.9778734), (1440.0, 809.7725613), (1500.0, 815.2792562), (1560.0, 
    820.5228938), (1620.0, 825.5255316), (1680.0, 830.3067617), (1740.0, 
    834.8840538), (1800.0, 839.2730404), (1860.0, 843.4877575), (1920.0, 
    847.5408468), (1980.0, 851.4437275), (2040.0, 855.2067436), (2100.0, 
    858.8392889), (2160.0, 862.349916), (2220.0, 865.746429), (2280.0, 
    869.0359653), (2340.0, 872.2250658), (2400.0, 875.3197372), (2460.0, 
    878.3255053), (2520.0, 881.2474635), (2580.0, 884.0903142), (2640.0, 
    886.8584059), (2700.0, 889.5557668), (2760.0, 892.1861333), (2820.0, 
    894.7529764), (2880.0, 897.2595254), (2940.0, 899.708788), (3000.0, 
    902.1035698), (3060.0, 904.4464908), (3120.0, 906.7400006), (3180.0, 
    908.9863921), (3240.0, 911.187814), (3300.0, 913.3462819), (3360.0, 
    915.4636885), (3420.0, 917.5418131), (3480.0, 919.5823298), (3540.0, 
    921.5868151), (3600.0, 923.5567552), (3660.0, 925.4935524), (3720.0, 
    927.3985305), (3780.0, 929.272941), (3840.0, 931.1179673), (3900.0, 
    932.9347297), (3960.0, 934.7242894), (4020.0, 936.4876525), (4080.0, 
    938.2257733), (4140.0, 939.9395579), (4200.0, 941.629867), (4260.0, 
    943.2975188), (4320.0, 944.9432918), (4380.0, 946.5679266), (4440.0, 
    948.1721291), (4500.0, 949.7565716), (4560.0, 951.3218955), (4620.0, 
    952.8687125), (4680.0, 954.3976068), (4740.0, 955.9091364), (4800.0, 
    957.4038344), (4860.0, 958.8822108), (4920.0, 960.3447534), (4980.0, 
    961.7919291), (5040.0, 963.224185), (5100.0, 964.6419498), (5160.0, 
    966.0456339), (5220.0, 967.4356315), (5280.0, 968.8123203), (5340.0, 
    970.1760633), (5400.0, 971.5272087), (5460.0, 972.8660913), (5520.0, 
    974.1930328), (5580.0, 975.5083426), (5640.0, 976.8123183), (5700.0, 
    978.1052462), (5760.0, 979.387402), (5820.0, 980.6590513), (5880.0, 
    981.9204499), (5940.0, 983.1718443), (6000.0, 984.4134723), (6060.0, 
    985.6455631), (6120.0, 986.8683381), (6180.0, 988.0820108), (6240.0, 
    989.2867873), (6300.0, 990.4828668), (6360.0, 991.6704417), (6420.0, 
    992.849698), (6480.0, 994.0208155), (6540.0, 995.1839681), (6600.0, 
    996.339324), (6660.0, 997.4870459), (6720.0, 998.6272916), (6780.0, 
    999.7602136), (6840.0, 1000.88596), (6900.0, 1002.004673), (6960.0, 
    1003.116492), (7020.0, 1004.221553), (7080.0, 1005.319984), (7140.0, 
    1006.411913), (7200.0, 1007.497462)), name='Fire-120', smooth=
    SOLVER_DEFAULT, timeSpan=STEP)
	
mdb.models['Model-1'].TabularAmplitude(data=((0.0, 20.0), (60.0, 332.5304157), 
    (120.0, 426.044443), (180.0, 487.12866), (240.0, 532.516808), (300.0, 
    568.4577615), (360.0, 598.0496271), (420.0, 623.0742538), (480.0, 
    644.6569038), (540.0, 663.5558175), (600.0, 680.3069656), (660.0, 
    695.303428), (720.0, 708.8420627), (780.0, 721.1524826), (840.0, 
    732.4158538), (900.0, 742.7775408), (960.0, 752.3558747), (1020.0, 
    761.248392), (1080.0, 769.5363737), (1140.0, 777.2882137), (1200.0, 
    784.5619596), (1260.0, 791.4072598), (1320.0, 797.8668765), (1380.0, 
    803.9778734), (1440.0, 809.7725613), (1500.0, 815.2792562), (1560.0, 
    820.5228938), (1620.0, 825.5255316), (1680.0, 830.3067617), (1740.0, 
    834.8840538), (1800.0, 839.2730404), (1860.0, 843.4877575), (1920.0, 
    847.5408468), (1980.0, 851.4437275), (2040.0, 855.2067436), (2100.0, 
    858.8392889), (2160.0, 862.349916), (2220.0, 865.746429), (2280.0, 
    869.0359653), (2340.0, 872.2250658), (2400.0, 875.3197372), (2460.0, 
    878.3255053), (2520.0, 881.2474635), (2580.0, 884.0903142), (2640.0, 
    886.8584059), (2700.0, 889.5557668), (2760.0, 892.1861333), (2820.0, 
    894.7529764), (2880.0, 897.2595254), (2940.0, 899.708788), (3000.0, 
    902.1035698), (3060.0, 904.4464908), (3120.0, 906.7400006), (3180.0, 
    908.9863921), (3240.0, 911.187814), (3300.0, 913.3462819), (3360.0, 
    915.4636885), (3420.0, 917.5418131), (3480.0, 919.5823298), (3540.0, 
    921.5868151), (3600.0, 923.5567552), (3660.0, 925.4935524), (3720.0, 
    927.3985305), (3780.0, 929.272941), (3840.0, 931.1179673), (3900.0, 
    932.9347297), (3960.0, 934.7242894), (4020.0, 936.4876525), (4080.0, 
    938.2257733), (4140.0, 939.9395579), (4200.0, 941.629867), (4260.0, 
    943.2975188), (4320.0, 944.9432918), (4380.0, 946.5679266), (4440.0, 
    948.1721291), (4500.0, 949.7565716), (4560.0, 951.3218955), (4620.0, 
    952.8687125), (4680.0, 954.3976068), (4740.0, 955.9091364), (4800.0, 
    957.4038344), (4860.0, 958.8822108), (4920.0, 960.3447534), (4980.0, 
    961.7919291), (5040.0, 963.224185), (5100.0, 964.6419498), (5160.0, 
    966.0456339), (5220.0, 967.4356315), (5280.0, 968.8123203), (5340.0, 
    970.1760633), (5400.0, 971.5272087), (5460.0, 972.8660913), (5520.0, 
    974.1930328), (5580.0, 975.5083426), (5640.0, 976.8123183), (5700.0, 
    978.1052462), (5760.0, 979.387402), (5820.0, 980.6590513), (5880.0, 
    981.9204499), (5940.0, 983.1718443), (6000.0, 984.4134723), (6060.0, 
    985.6455631), (6120.0, 986.8683381), (6180.0, 988.0820108), (6240.0, 
    989.2867873), (6300.0, 990.4828668), (6360.0, 991.6704417), (6420.0, 
    992.849698), (6480.0, 994.0208155), (6540.0, 995.1839681), (6600.0, 
    996.339324), (6660.0, 997.4870459), (6720.0, 998.6272916), (6780.0, 
    999.7602136), (6840.0, 1000.88596), (6900.0, 1002.004673), (6960.0, 
    1003.116492), (7020.0, 1004.221553), (7080.0, 1005.319984), (7140.0, 
    1006.411913), (7200.0, 1007.497462), (7260.0, 1008.576751), (7320.0, 
    1009.649895), (7380.0, 1010.717007), (7440.0, 1011.778196), (7500.0, 
    1012.833568), (7560.0, 1013.883225), (7620.0, 1014.927268), (7680.0, 
    1015.965794), (7740.0, 1016.998898), (7800.0, 1018.026672), (7860.0, 
    1019.049206), (7920.0, 1020.066586), (7980.0, 1021.078898), (8040.0, 
    1022.086223), (8100.0, 1023.088643), (8160.0, 1024.086236), (8220.0, 
    1025.079079), (8280.0, 1026.067244), (8340.0, 1027.050806), (8400.0, 
    1028.029835), (8460.0, 1029.004399), (8520.0, 1029.974566), (8580.0, 
    1030.940401), (8640.0, 1031.901968), (8700.0, 1032.859331), (8760.0, 
    1033.812549), (8820.0, 1034.761682), (8880.0, 1035.706788), (8940.0, 
    1036.647924), (9000.0, 1037.585145)), name='Fire-150', smooth=
    SOLVER_DEFAULT, timeSpan=STEP)
	
mdb.models['Model-1'].TabularAmplitude(data=((0.0, 20.0), (60.0, 332.5304157), 
    (120.0, 426.044443), (180.0, 487.12866), (240.0, 532.516808), (300.0, 
    568.4577615), (360.0, 598.0496271), (420.0, 623.0742538), (480.0, 
    644.6569038), (540.0, 663.5558175), (600.0, 680.3069656), (660.0, 
    695.303428), (720.0, 708.8420627), (780.0, 721.1524826), (840.0, 
    732.4158538), (900.0, 742.7775408), (960.0, 752.3558747), (1020.0, 
    761.248392), (1080.0, 769.5363737), (1140.0, 777.2882137), (1200.0, 
    784.5619596), (1260.0, 791.4072598), (1320.0, 797.8668765), (1380.0, 
    803.9778734), (1440.0, 809.7725613), (1500.0, 815.2792562), (1560.0, 
    820.5228938), (1620.0, 825.5255316), (1680.0, 830.3067617), (1740.0, 
    834.8840538), (1800.0, 839.2730404), (1860.0, 843.4877575), (1920.0, 
    847.5408468), (1980.0, 851.4437275), (2040.0, 855.2067436), (2100.0, 
    858.8392889), (2160.0, 862.349916), (2220.0, 865.746429), (2280.0, 
    869.0359653), (2340.0, 872.2250658), (2400.0, 875.3197372), (2460.0, 
    878.3255053), (2520.0, 881.2474635), (2580.0, 884.0903142), (2640.0, 
    886.8584059), (2700.0, 889.5557668), (2760.0, 892.1861333), (2820.0, 
    894.7529764), (2880.0, 897.2595254), (2940.0, 899.708788), (3000.0, 
    902.1035698), (3060.0, 904.4464908), (3120.0, 906.7400006), (3180.0, 
    908.9863921), (3240.0, 911.187814), (3300.0, 913.3462819), (3360.0, 
    915.4636885), (3420.0, 917.5418131), (3480.0, 919.5823298), (3540.0, 
    921.5868151), (3600.0, 923.5567552), (3660.0, 925.4935524), (3720.0, 
    927.3985305), (3780.0, 929.272941), (3840.0, 931.1179673), (3900.0, 
    932.9347297), (3960.0, 934.7242894), (4020.0, 936.4876525), (4080.0, 
    938.2257733), (4140.0, 939.9395579), (4200.0, 941.629867), (4260.0, 
    943.2975188), (4320.0, 944.9432918), (4380.0, 946.5679266), (4440.0, 
    948.1721291), (4500.0, 949.7565716), (4560.0, 951.3218955), (4620.0, 
    952.8687125), (4680.0, 954.3976068), (4740.0, 955.9091364), (4800.0, 
    957.4038344), (4860.0, 958.8822108), (4920.0, 960.3447534), (4980.0, 
    961.7919291), (5040.0, 963.224185), (5100.0, 964.6419498), (5160.0, 
    966.0456339), (5220.0, 967.4356315), (5280.0, 968.8123203), (5340.0, 
    970.1760633), (5400.0, 971.5272087), (5460.0, 972.8660913), (5520.0, 
    974.1930328), (5580.0, 975.5083426), (5640.0, 976.8123183), (5700.0, 
    978.1052462), (5760.0, 979.387402), (5820.0, 980.6590513), (5880.0, 
    981.9204499), (5940.0, 983.1718443), (6000.0, 984.4134723), (6060.0, 
    985.6455631), (6120.0, 986.8683381), (6180.0, 988.0820108), (6240.0, 
    989.2867873), (6300.0, 990.4828668), (6360.0, 991.6704417), (6420.0, 
    992.849698), (6480.0, 994.0208155), (6540.0, 995.1839681), (6600.0, 
    996.339324), (6660.0, 997.4870459), (6720.0, 998.6272916), (6780.0, 
    999.7602136), (6840.0, 1000.88596), (6900.0, 1002.004673), (6960.0, 
    1003.116492), (7020.0, 1004.221553), (7080.0, 1005.319984), (7140.0, 
    1006.411913), (7200.0, 1007.497462), (7260.0, 1008.576751), (7320.0, 
    1009.649895), (7380.0, 1010.717007), (7440.0, 1011.778196), (7500.0, 
    1012.833568), (7560.0, 1013.883225), (7620.0, 1014.927268), (7680.0, 
    1015.965794), (7740.0, 1016.998898), (7800.0, 1018.026672), (7860.0, 
    1019.049206), (7920.0, 1020.066586), (7980.0, 1021.078898), (8040.0, 
    1022.086223), (8100.0, 1023.088643), (8160.0, 1024.086236), (8220.0, 
    1025.079079), (8280.0, 1026.067244), (8340.0, 1027.050806), (8400.0, 
    1028.029835), (8460.0, 1029.004399), (8520.0, 1029.974566), (8580.0, 
    1030.940401), (8640.0, 1031.901968), (8700.0, 1032.859331), (8760.0, 
    1033.812549), (8820.0, 1034.761682), (8880.0, 1035.706788), (8940.0, 
    1036.647924), (9000.0, 1037.585145), (9060.0, 1038.518505), (9120.0, 
    1039.448058), (9180.0, 1040.373854), (9240.0, 1041.295944), (9300.0, 
    1042.214379), (9360.0, 1043.129205), (9420.0, 1044.04047), (9480.0, 
    1044.948221), (9540.0, 1045.852502), (9600.0, 1046.753358), (9660.0, 
    1047.650832), (9720.0, 1048.544966), (9780.0, 1049.435802), (9840.0, 
    1050.32338), (9900.0, 1051.20774), (9960.0, 1052.088921), (10020.0, 
    1052.966961), (10080.0, 1053.841898), (10140.0, 1054.713767), (10200.0, 
    1055.582606), (10260.0, 1056.448448), (10320.0, 1057.311329), (10380.0, 
    1058.171282), (10440.0, 1059.028341), (10500.0, 1059.882537), (10560.0, 
    1060.733903), (10620.0, 1061.58247), (10680.0, 1062.428268), (10740.0, 
    1063.271329), (10800.0, 1064.11168)), name='Fire-180', smooth=
    SOLVER_DEFAULT, timeSpan=STEP)
	
mdb.models['Model-1'].TabularAmplitude(data=((0.0, 20.0), (60.0, 332.5304157), 
    (120.0, 426.044443), (180.0, 487.12866), (240.0, 532.516808), (300.0, 
    568.4577615), (360.0, 598.0496271), (420.0, 623.0742538), (480.0, 
    644.6569038), (540.0, 663.5558175), (600.0, 680.3069656), (660.0, 
    695.303428), (720.0, 708.8420627), (780.0, 721.1524826), (840.0, 
    732.4158538), (900.0, 742.7775408), (960.0, 752.3558747), (1020.0, 
    761.248392), (1080.0, 769.5363737), (1140.0, 777.2882137), (1200.0, 
    784.5619596), (1260.0, 791.4072598), (1320.0, 797.8668765), (1380.0, 
    803.9778734), (1440.0, 809.7725613), (1500.0, 815.2792562), (1560.0, 
    820.5228938), (1620.0, 825.5255316), (1680.0, 830.3067617), (1740.0, 
    834.8840538), (1800.0, 839.2730404), (1860.0, 843.4877575), (1920.0, 
    847.5408468), (1980.0, 851.4437275), (2040.0, 855.2067436), (2100.0, 
    858.8392889), (2160.0, 862.349916), (2220.0, 865.746429), (2280.0, 
    869.0359653), (2340.0, 872.2250658), (2400.0, 875.3197372), (2460.0, 
    878.3255053), (2520.0, 881.2474635), (2580.0, 884.0903142), (2640.0, 
    886.8584059), (2700.0, 889.5557668), (2760.0, 892.1861333), (2820.0, 
    894.7529764), (2880.0, 897.2595254), (2940.0, 899.708788), (3000.0, 
    902.1035698), (3060.0, 904.4464908), (3120.0, 906.7400006), (3180.0, 
    908.9863921), (3240.0, 911.187814), (3300.0, 913.3462819), (3360.0, 
    915.4636885), (3420.0, 917.5418131), (3480.0, 919.5823298), (3540.0, 
    921.5868151), (3600.0, 923.5567552), (3660.0, 925.4935524), (3720.0, 
    927.3985305), (3780.0, 929.272941), (3840.0, 931.1179673), (3900.0, 
    932.9347297), (3960.0, 934.7242894), (4020.0, 936.4876525), (4080.0, 
    938.2257733), (4140.0, 939.9395579), (4200.0, 941.629867), (4260.0, 
    943.2975188), (4320.0, 944.9432918), (4380.0, 946.5679266), (4440.0, 
    948.1721291), (4500.0, 949.7565716), (4560.0, 951.3218955), (4620.0, 
    952.8687125), (4680.0, 954.3976068), (4740.0, 955.9091364), (4800.0, 
    957.4038344), (4860.0, 958.8822108), (4920.0, 960.3447534), (4980.0, 
    961.7919291), (5040.0, 963.224185), (5100.0, 964.6419498), (5160.0, 
    966.0456339), (5220.0, 967.4356315), (5280.0, 968.8123203), (5340.0, 
    970.1760633), (5400.0, 971.5272087), (5460.0, 972.8660913), (5520.0, 
    974.1930328), (5580.0, 975.5083426), (5640.0, 976.8123183), (5700.0, 
    978.1052462), (5760.0, 979.387402), (5820.0, 980.6590513), (5880.0, 
    981.9204499), (5940.0, 983.1718443), (6000.0, 984.4134723), (6060.0, 
    985.6455631), (6120.0, 986.8683381), (6180.0, 988.0820108), (6240.0, 
    989.2867873), (6300.0, 990.4828668), (6360.0, 991.6704417), (6420.0, 
    992.849698), (6480.0, 994.0208155), (6540.0, 995.1839681), (6600.0, 
    996.339324), (6660.0, 997.4870459), (6720.0, 998.6272916), (6780.0, 
    999.7602136), (6840.0, 1000.88596), (6900.0, 1002.004673), (6960.0, 
    1003.116492), (7020.0, 1004.221553), (7080.0, 1005.319984), (7140.0, 
    1006.411913), (7200.0, 1007.497462), (7260.0, 1008.576751), (7320.0, 
    1009.649895), (7380.0, 1010.717007), (7440.0, 1011.778196), (7500.0, 
    1012.833568), (7560.0, 1013.883225), (7620.0, 1014.927268), (7680.0, 
    1015.965794), (7740.0, 1016.998898), (7800.0, 1018.026672), (7860.0, 
    1019.049206), (7920.0, 1020.066586), (7980.0, 1021.078898), (8040.0, 
    1022.086223), (8100.0, 1023.088643), (8160.0, 1024.086236), (8220.0, 
    1025.079079), (8280.0, 1026.067244), (8340.0, 1027.050806), (8400.0, 
    1028.029835), (8460.0, 1029.004399), (8520.0, 1029.974566), (8580.0, 
    1030.940401), (8640.0, 1031.901968), (8700.0, 1032.859331), (8760.0, 
    1033.812549), (8820.0, 1034.761682), (8880.0, 1035.706788), (8940.0, 
    1036.647924), (9000.0, 1037.585145), (9060.0, 1038.518505), (9120.0, 
    1039.448058), (9180.0, 1040.373854), (9240.0, 1041.295944), (9300.0, 
    1042.214379), (9360.0, 1043.129205), (9420.0, 1044.04047), (9480.0, 
    1044.948221), (9540.0, 1045.852502), (9600.0, 1046.753358), (9660.0, 
    1047.650832), (9720.0, 1048.544966), (9780.0, 1049.435802), (9840.0, 
    1050.32338), (9900.0, 1051.20774), (9960.0, 1052.088921), (10020.0, 
    1052.966961), (10080.0, 1053.841898), (10140.0, 1054.713767), (10200.0, 
    1055.582606), (10260.0, 1056.448448), (10320.0, 1057.311329), (10380.0, 
    1058.171282), (10440.0, 1059.028341), (10500.0, 1059.882537), (10560.0, 
    1060.733903), (10620.0, 1061.58247), (10680.0, 1062.428268), (10740.0, 
    1063.271329), (10800.0, 1064.11168), (10860.0, 1064.949352), (10920.0, 
    1065.784372), (10980.0, 1066.616769), (11040.0, 1067.44657), (11100.0, 
    1068.273802), (11160.0, 1069.098491), (11220.0, 1069.920664), (11280.0, 
    1070.740345), (11340.0, 1071.55756), (11400.0, 1072.372334), (11460.0, 
    1073.184691), (11520.0, 1073.994654), (11580.0, 1074.802247), (11640.0, 
    1075.607493), (11700.0, 1076.410414), (11760.0, 1077.211033), (11820.0, 
    1078.009372), (11880.0, 1078.805452), (11940.0, 1079.599294), (12000.0, 
    1080.390919), (12060.0, 1081.180347), (12120.0, 1081.967599), (12180.0, 
    1082.752694), (12240.0, 1083.535652), (12300.0, 1084.316493), (12360.0, 
    1085.095234), (12420.0, 1085.871895), (12480.0, 1086.646493), (12540.0, 
    1087.419047), (12600.0, 1088.189575), (12660.0, 1088.958094), (12720.0, 
    1089.724622), (12780.0, 1090.489174), (12840.0, 1091.251768), (12900.0, 
    1092.012421), (12960.0, 1092.771148), (13020.0, 1093.527966), (13080.0, 
    1094.28289), (13140.0, 1095.035936), (13200.0, 1095.787118), (13260.0, 
    1096.536453), (13320.0, 1097.283955), (13380.0, 1098.029638), (13440.0, 
    1098.773518), (13500.0, 1099.515608), (13560.0, 1100.255922), (13620.0, 
    1100.994474), (13680.0, 1101.731278), (13740.0, 1102.466347), (13800.0, 
    1103.199695), (13860.0, 1103.931334), (13920.0, 1104.661278), (13980.0, 
    1105.38954), (14040.0, 1106.116131), (14100.0, 1106.841064), (14160.0, 
    1107.564353), (14220.0, 1108.286007), (14280.0, 1109.006041), (14340.0, 
    1109.724465), (14400.0, 1110.441291)), name='Fire-240', smooth=
    SOLVER_DEFAULT, timeSpan=STEP)

#Flim_Condition

mdb.models['Model-1'].FilmConditionProp(dependencies=0, name='F', property=((0.025, ), ), temperatureDependency=OFF)
mdb.models['Model-1'].FilmConditionProp(dependencies=0, name='NF', property=((0.008, ), ), temperatureDependency=OFF)

#Surface_Create

mdb.models['Model-1'].rootAssembly.Surface(name='F-1', side1Faces=mdb.models['Model-1'].rootAssembly.instances['Beam-1'].faces.findAt(((0.5*C1, -0.5*D, 0.5*H), ), ((-0.5*C1, -0.5*D, 0.5*H), ), ((0.5*C+C1, -0.5*D, 0.5*H), ), ((-0.5*C-C1, -0.5*D, 0.5*H), ), ))
mdb.models['Model-1'].rootAssembly.Surface(name='F-2', side1Faces=mdb.models['Model-1'].rootAssembly.instances['Beam-1'].faces.findAt(((-0.5*W, 0.5*C2, 0.5*H), ), ((-0.5*W, -0.5*C2, 0.5*H), ), ((-0.5*W, 0.5*C+C2, 0.5*H), ), ((-0.5*W, -0.5*C-C2, 0.5*H), ), ((0.5*C1, -0.5*D, 0.5*H), ), ((-0.5*C1, -0.5*D, 0.5*H), ), ((0.5*C+C1, -0.5*D, 0.5*H), ), ((-0.5*C-C1, -0.5*D, 0.5*H), ), ))
mdb.models['Model-1'].rootAssembly.Surface(name='F-3', side1Faces=mdb.models['Model-1'].rootAssembly.instances['Beam-1'].faces.findAt(((0.5*W, 0.5*C2, 0.5*H), ), ((0.5*W, -0.5*C2, 0.5*H), ), ((0.5*W, 0.5*C+C2, 0.5*H), ), ((0.5*W, -0.5*C-C2, 0.5*H), ), ((0.5*C1, -0.5*D, 0.5*H), ), ((-0.5*C1, -0.5*D, 0.5*H), ), ((0.5*C+C1, -0.5*D, 0.5*H), ), ((-0.5*C-C1, -0.5*D, 0.5*H), ), ((-0.5*W, 0.5*C2, 0.5*H), ), ((-0.5*W, -0.5*C2, 0.5*H), ), ((-0.5*W, 0.5*C+C2, 0.5*H), ), ((-0.5*W, -0.5*C-C2, 0.5*H), ), ))
mdb.models['Model-1'].rootAssembly.Surface(name='F-4', side1Faces=mdb.models['Model-1'].rootAssembly.instances['Beam-1'].faces.findAt(((0.5*W, 0.5*C2, 0.5*H), ), ((0.5*W, -0.5*C2, 0.5*H), ), ((0.5*W, 0.5*C+C2, 0.5*H), ), ((0.5*W, -0.5*C-C2, 0.5*H), ), ((0.5*C1, 0.5*D, 0.5*H), ), ((-0.5*C1, 0.5*D, 0.5*H), ), ((0.5*C+C1, 0.5*D, 0.5*H), ), ((-0.5*C-C1, 0.5*D, 0.5*H), ), ((-0.5*W, 0.5*C2, 0.5*H), ), ((-0.5*W, -0.5*C2, 0.5*H), ), ((-0.5*W, 0.5*C+C2, 0.5*H), ), ((-0.5*W, -0.5*C-C2, 0.5*H), ), ((0.5*C1, -0.5*D, 0.5*H), ), ((-0.5*C1, -0.5*D, 0.5*H), ), ((0.5*C+C1, -0.5*D, 0.5*H), ), ((-0.5*C-C1, -0.5*D, 0.5*H), ), ))
mdb.models['Model-1'].rootAssembly.Surface(name='NF-1', side1Faces=mdb.models['Model-1'].rootAssembly.instances['Beam-1'].faces.findAt(((0.5*C1, 0.5*C2, H), ), ((-0.5*C1, 0.5*C2, H), ), ((0.5*C1, -0.5*C2, H), ), ((-0.5*C1, -0.5*C2, H), ), ((0.5*C+C1, 0.5*C2, H), ), ((-0.5*C-C1, 0.5*C2, H), ), ((0.5*C+C1, -0.5*C2, H), ), ((-0.5*C-C1, -0.5*C2, H), ), ((0.5*C1, 0.5*C+C2, H), ), ((-0.5*C1, 0.5*C+C2, H), ), ((0.5*C1, -0.5*C-C2, H), ), ((-0.5*C1, -0.5*C-C2, H), ), ((0.5*C+C1, 0.5*C+C2, H), ), ((-0.5*C-C1, 0.5*C+C2, H), ), ((0.5*C+C1, -0.5*C-C2, H), ), ((-0.5*C-C1, -0.5*C-C2, H), ), ((0.5*C1, 0.5*C2, 0), ), ((-0.5*C1, 0.5*C2, 0), ), ((0.5*C1, -0.5*C2, 0), ), ((-0.5*C1, -0.5*C2, 0), ), ((0.5*C+C1, 0.5*C2, 0), ), ((-0.5*C-C1, 0.5*C2, 0), ), ((0.5*C+C1, -0.5*C2, 0), ), ((-0.5*C-C1, -0.5*C2, 0), ), ((0.5*C1, 0.5*C+C2, 0), ), ((-0.5*C1, 0.5*C+C2, 0), ), ((0.5*C1, -0.5*C-C2, 0), ), ((-0.5*C1, -0.5*C-C2, 0), ), ((0.5*C+C1, 0.5*C+C2, 0), ), ((-0.5*C-C1, 0.5*C+C2, 0), ), ((0.5*C+C1, -0.5*C-C2, 0), ), ((-0.5*C-C1, -0.5*C-C2, 0), ), ((0.5*W, 0.5*C2, 0.5*H), ), ((0.5*W, -0.5*C2, 0.5*H), ), ((0.5*W, 0.5*C+C2, 0.5*H), ), ((0.5*W, -0.5*C-C2, 0.5*H), ), ((0.5*C1, 0.5*D, 0.5*H), ), ((-0.5*C1, 0.5*D, 0.5*H), ), ((0.5*C+C1, 0.5*D, 0.5*H), ), ((-0.5*C-C1, 0.5*D, 0.5*H), ), ((-0.5*W, 0.5*C2, 0.5*H), ), ((-0.5*W, -0.5*C2, 0.5*H), ), ((-0.5*W, 0.5*C+C2, 0.5*H), ), ((-0.5*W, -0.5*C-C2, 0.5*H), ), ))
mdb.models['Model-1'].rootAssembly.Surface(name='NF-2', side1Faces=mdb.models['Model-1'].rootAssembly.instances['Beam-1'].faces.findAt(((0.5*C1, 0.5*C2, H), ), ((-0.5*C1, 0.5*C2, H), ), ((0.5*C1, -0.5*C2, H), ), ((-0.5*C1, -0.5*C2, H), ), ((0.5*C+C1, 0.5*C2, H), ), ((-0.5*C-C1, 0.5*C2, H), ), ((0.5*C+C1, -0.5*C2, H), ), ((-0.5*C-C1, -0.5*C2, H), ), ((0.5*C1, 0.5*C+C2, H), ), ((-0.5*C1, 0.5*C+C2, H), ), ((0.5*C1, -0.5*C-C2, H), ), ((-0.5*C1, -0.5*C-C2, H), ), ((0.5*C+C1, 0.5*C+C2, H), ), ((-0.5*C-C1, 0.5*C+C2, H), ), ((0.5*C+C1, -0.5*C-C2, H), ), ((-0.5*C-C1, -0.5*C-C2, H), ), ((0.5*C1, 0.5*C2, 0), ), ((-0.5*C1, 0.5*C2, 0), ), ((0.5*C1, -0.5*C2, 0), ), ((-0.5*C1, -0.5*C2, 0), ), ((0.5*C+C1, 0.5*C2, 0), ), ((-0.5*C-C1, 0.5*C2, 0), ), ((0.5*C+C1, -0.5*C2, 0), ), ((-0.5*C-C1, -0.5*C2, 0), ), ((0.5*C1, 0.5*C+C2, 0), ), ((-0.5*C1, 0.5*C+C2, 0), ), ((0.5*C1, -0.5*C-C2, 0), ), ((-0.5*C1, -0.5*C-C2, 0), ), ((0.5*C+C1, 0.5*C+C2, 0), ), ((-0.5*C-C1, 0.5*C+C2, 0), ), ((0.5*C+C1, -0.5*C-C2, 0), ), ((-0.5*C-C1, -0.5*C-C2, 0), ), ((0.5*W, 0.5*C2, 0.5*H), ), ((0.5*W, -0.5*C2, 0.5*H), ), ((0.5*W, 0.5*C+C2, 0.5*H), ), ((0.5*W, -0.5*C-C2, 0.5*H), ), ((0.5*C1, 0.5*D, 0.5*H), ), ((-0.5*C1, 0.5*D, 0.5*H), ), ((0.5*C+C1, 0.5*D, 0.5*H), ), ((-0.5*C-C1, 0.5*D, 0.5*H), ), ))
mdb.models['Model-1'].rootAssembly.Surface(name='NF-3', side1Faces=mdb.models['Model-1'].rootAssembly.instances['Beam-1'].faces.findAt(((0.5*C1, 0.5*C2, H), ), ((-0.5*C1, 0.5*C2, H), ), ((0.5*C1, -0.5*C2, H), ), ((-0.5*C1, -0.5*C2, H), ), ((0.5*C+C1, 0.5*C2, H), ), ((-0.5*C-C1, 0.5*C2, H), ), ((0.5*C+C1, -0.5*C2, H), ), ((-0.5*C-C1, -0.5*C2, H), ), ((0.5*C1, 0.5*C+C2, H), ), ((-0.5*C1, 0.5*C+C2, H), ), ((0.5*C1, -0.5*C-C2, H), ), ((-0.5*C1, -0.5*C-C2, H), ), ((0.5*C+C1, 0.5*C+C2, H), ), ((-0.5*C-C1, 0.5*C+C2, H), ), ((0.5*C+C1, -0.5*C-C2, H), ), ((-0.5*C-C1, -0.5*C-C2, H), ), ((0.5*C1, 0.5*C2, 0), ), ((-0.5*C1, 0.5*C2, 0), ), ((0.5*C1, -0.5*C2, 0), ), ((-0.5*C1, -0.5*C2, 0), ), ((0.5*C+C1, 0.5*C2, 0), ), ((-0.5*C-C1, 0.5*C2, 0), ), ((0.5*C+C1, -0.5*C2, 0), ), ((-0.5*C-C1, -0.5*C2, 0), ), ((0.5*C1, 0.5*C+C2, 0), ), ((-0.5*C1, 0.5*C+C2, 0), ), ((0.5*C1, -0.5*C-C2, 0), ), ((-0.5*C1, -0.5*C-C2, 0), ), ((0.5*C+C1, 0.5*C+C2, 0), ), ((-0.5*C-C1, 0.5*C+C2, 0), ), ((0.5*C+C1, -0.5*C-C2, 0), ), ((-0.5*C-C1, -0.5*C-C2, 0), ), ((0.5*C1, 0.5*D, 0.5*H), ), ((-0.5*C1, 0.5*D, 0.5*H), ), ((0.5*C+C1, 0.5*D, 0.5*H), ), ((-0.5*C-C1, 0.5*D, 0.5*H), ), ))
mdb.models['Model-1'].rootAssembly.Surface(name='NF-4', side1Faces=mdb.models['Model-1'].rootAssembly.instances['Beam-1'].faces.findAt(((0.5*C1, 0.5*C2, H), ), ((-0.5*C1, 0.5*C2, H), ), ((0.5*C1, -0.5*C2, H), ), ((-0.5*C1, -0.5*C2, H), ), ((0.5*C+C1, 0.5*C2, H), ), ((-0.5*C-C1, 0.5*C2, H), ), ((0.5*C+C1, -0.5*C2, H), ), ((-0.5*C-C1, -0.5*C2, H), ), ((0.5*C1, 0.5*C+C2, H), ), ((-0.5*C1, 0.5*C+C2, H), ), ((0.5*C1, -0.5*C-C2, H), ), ((-0.5*C1, -0.5*C-C2, H), ), ((0.5*C+C1, 0.5*C+C2, H), ), ((-0.5*C-C1, 0.5*C+C2, H), ), ((0.5*C+C1, -0.5*C-C2, H), ), ((-0.5*C-C1, -0.5*C-C2, H), ), ((0.5*C1, 0.5*C2, 0), ), ((-0.5*C1, 0.5*C2, 0), ), ((0.5*C1, -0.5*C2, 0), ), ((-0.5*C1, -0.5*C2, 0), ), ((0.5*C+C1, 0.5*C2, 0), ), ((-0.5*C-C1, 0.5*C2, 0), ), ((0.5*C+C1, -0.5*C2, 0), ), ((-0.5*C-C1, -0.5*C2, 0), ), ((0.5*C1, 0.5*C+C2, 0), ), ((-0.5*C1, 0.5*C+C2, 0), ), ((0.5*C1, -0.5*C-C2, 0), ), ((-0.5*C1, -0.5*C-C2, 0), ), ((0.5*C+C1, 0.5*C+C2, 0), ), ((-0.5*C-C1, 0.5*C+C2, 0), ), ((0.5*C+C1, -0.5*C-C2, 0), ), ((-0.5*C-C1, -0.5*C-C2, 0), ), ))

#Interaction_Tie
#Change_If_Need
mdb.models['Model-1'].FilmCondition(createStepName='Fire', definition=
    PROPERTY_REF, interactionProperty='F', name='F', sinkAmplitude='Fire-240', 
    sinkDistributionType=UNIFORM, sinkFieldName='', sinkTemperature=1.0, 
    surface=mdb.models['Model-1'].rootAssembly.surfaces['F-4'])
mdb.models['Model-1'].FilmCondition(createStepName='Fire', definition=
    PROPERTY_REF, interactionProperty='NF', name='NF', sinkAmplitude='Fire-240'
    , sinkDistributionType=UNIFORM, sinkFieldName='', sinkTemperature=1.0, 
    surface=mdb.models['Model-1'].rootAssembly.surfaces['NF-4'])
mdb.models['Model-1'].RadiationToAmbient(ambientTemperature=1.0, 
    ambientTemperatureAmp='Fire-240', createStepName='Fire', distributionType=
    UNIFORM, emissivity=0.8, field='', name='FR', radiationType=AMBIENT, 
    surface=mdb.models['Model-1'].rootAssembly.surfaces['F-4'])

mdb.models['Model-1'].rootAssembly.Set(edges=
    mdb.models['Model-1'].rootAssembly.instances['Beam-1'].edges.findAt(((
    C1, -C2, 0.75*H), )), name='B1')
mdb.models['Model-1'].rootAssembly.Set(edges=
    mdb.models['Model-1'].rootAssembly.instances['Beam-1'].edges.findAt(((
    C1, C2, 0.75*H), )), name='B3')
mdb.models['Model-1'].rootAssembly.Set(edges=
    mdb.models['Model-1'].rootAssembly.instances['Beam-1'].edges.findAt(((
    -C1, C2, 0.75*H), )), name='B5')
mdb.models['Model-1'].rootAssembly.Set(edges=
    mdb.models['Model-1'].rootAssembly.instances['Beam-1'].edges.findAt(((
    -C1, -C2, 0.75*H), )), name='B7')
mdb.models['Model-1'].rootAssembly.Set(edges=
    mdb.models['Model-1'].rootAssembly.instances['Bar-1-lin-2-1'].edges.findAt(
    ((C1, -C2, 0.25*H), )), name='B2')
mdb.models['Model-1'].rootAssembly.Set(edges=
    mdb.models['Model-1'].rootAssembly.instances['Bar-1-lin-1-2-lin-2-1'].edges.findAt(
    ((C1, C2, 0.25*H), )), name='B4')
mdb.models['Model-1'].rootAssembly.Set(edges=
    mdb.models['Model-1'].rootAssembly.instances['Bar-1-lin-1-2'].edges.findAt(
    ((-C1, C2, 0.25*H), )), name='B6')
mdb.models['Model-1'].rootAssembly.Set(edges=
    mdb.models['Model-1'].rootAssembly.instances['Bar-1'].edges.findAt(((
    -C1, -C2, 0.25*H), )), name='B8')

mdb.models['Model-1'].Tie(adjust=ON, master=
    mdb.models['Model-1'].rootAssembly.sets['B1'], name='T1', 
    positionToleranceMethod=COMPUTED, slave=
    mdb.models['Model-1'].rootAssembly.sets['B2'], thickness=ON, tieRotations=
    ON)

mdb.models['Model-1'].Tie(adjust=ON, master=
    mdb.models['Model-1'].rootAssembly.sets['B3'], name='T2', 
    positionToleranceMethod=COMPUTED, slave=
    mdb.models['Model-1'].rootAssembly.sets['B4'], thickness=ON, tieRotations=
    ON)

mdb.models['Model-1'].Tie(adjust=ON, master=
    mdb.models['Model-1'].rootAssembly.sets['B5'], name='T3', 
    positionToleranceMethod=COMPUTED, slave=
    mdb.models['Model-1'].rootAssembly.sets['B6'], thickness=ON, tieRotations=
    ON)

mdb.models['Model-1'].Tie(adjust=ON, master=
    mdb.models['Model-1'].rootAssembly.sets['B7'], name='T4', 
    positionToleranceMethod=COMPUTED, slave=
    mdb.models['Model-1'].rootAssembly.sets['B8'], thickness=ON, tieRotations=
    ON)
    
    
#Mesh

mdb.models['Model-1'].parts['Bar'].seedPart(deviationFactor=0.1, minSizeFactor=0.1, size=d)
mdb.models['Model-1'].parts['Bar'].generateMesh()
mdb.models['Model-1'].parts['Bar'].setElementType(elemTypes=(ElemType(elemCode=DC1D2, elemLibrary=STANDARD), ), regions=(mdb.models['Model-1'].parts['Bar'].edges.findAt(((0.5*H, 0.0, 0.0), )), ))

mdb.models['Model-1'].parts['Beam'].seedPart(deviationFactor=0.1, minSizeFactor=0.1, size=d)
mdb.models['Model-1'].parts['Beam'].generateMesh()
mdb.models['Model-1'].parts['Beam'].setElementType(elemTypes=(ElemType(elemCode=DC3D8, elemLibrary=STANDARD), ElemType(elemCode=DC3D6, elemLibrary=STANDARD), ElemType(elemCode=DC3D4, elemLibrary=STANDARD)), regions=(mdb.models['Model-1'].parts['Beam'].cells.findAt(((0.5*C1, 0.5*C2, H), ), ((-0.5*C1, 0.5*C2, H), ), ((0.5*C1, -0.5*C2, H), ), ((-0.5*C1, -0.5*C2, H), ), ((0.5*C1, 0.5*D, H), ), ((0.5*C1, -0.5*D, H), ), ((-0.5*C1, 0.5*D, H), ), ((-0.5*C1, -0.5*D, H), ), ((1.05*C1, 0.5*C2, H), ), ((1.05*C1, -0.5*C2, H), ), ((-1.05*C1, 0.5*C2, H), ), ((-1.05*C1, -0.5*C2, H), ), ((0.45*W, 0.45*D, H), ), ((0.45*W, -0.45*D, H), ), ((-0.45*W, 0.45*D, H), ), ((-0.45*W, -0.45*D, H), ), ), ))
mdb.models['Model-1'].rootAssembly.regenerate()

#Interaction_Node

mdb.models['Model-1'].rootAssembly.DatumPointByCoordinate(coords=(0.0, 0.0, 0.5*H))
mdb.models['Model-1'].rootAssembly.DatumPointByCoordinate(coords=(-C1, C2, 0.5*H))
mdb.models['Model-1'].rootAssembly.DatumPointByCoordinate(coords=(C1, C2, 0.5*H))
mdb.models['Model-1'].rootAssembly.DatumPointByCoordinate(coords=(-C1, -C2, 0.5*H))
mdb.models['Model-1'].rootAssembly.DatumPointByCoordinate(coords=(C1, -C2, 0.5*H))

#nodes_[element:node]

#mdb.models['Model-1'].rootAssembly.Set(name='TC5-C', nodes=mdb.models['Model-1'].rootAssembly.instances['Beam-1'].nodes[1843:1844])
mdb.models['Model-1'].rootAssembly.Set(name='TC1-B1', nodes=mdb.models['Model-1'].rootAssembly.instances['Bar-1-lin-2-1'].nodes[b:b+1])
mdb.models['Model-1'].rootAssembly.Set(name='TC1-B2', nodes=mdb.models['Model-1'].rootAssembly.instances['Bar-1-lin-1-2-lin-2-1'].nodes[b:b+1])
mdb.models['Model-1'].rootAssembly.Set(name='TC1-B3', nodes=mdb.models['Model-1'].rootAssembly.instances['Bar-1-lin-1-2'].nodes[b:b+1])
mdb.models['Model-1'].rootAssembly.Set(name='TC1-B4', nodes=mdb.models['Model-1'].rootAssembly.instances['Bar-1'].nodes[b:b+1])

#Multiple_Model

mdb.Model(name='Model-2', objectToCopy=mdb.models['Model-1'])
mdb.models['Model-2'].interactions['F'].setValues(definition=PROPERTY_REF, 
    interactionProperty='F', sinkAmplitude='Fire-240', sinkTemperature=1.0, 
    surface=mdb.models['Model-2'].rootAssembly.surfaces['F-3'])
mdb.models['Model-2'].interactions['FR'].setValues(ambientTemperature=1.0, 
    ambientTemperatureAmp='Fire-240', distributionType=UNIFORM, emissivity=0.8, 
    field='', radiationType=AMBIENT, surface=
    mdb.models['Model-2'].rootAssembly.surfaces['F-3'])
mdb.models['Model-2'].interactions['NF'].setValues(definition=PROPERTY_REF, 
    interactionProperty='NF', sinkAmplitude='Fire-240', sinkTemperature=1.0, 
    surface=mdb.models['Model-2'].rootAssembly.surfaces['NF-3'])
	
mdb.Model(name='Model-3', objectToCopy=mdb.models['Model-2'])
mdb.models['Model-3'].interactions['F'].setValues(definition=PROPERTY_REF, 
    interactionProperty='F', sinkAmplitude='Fire-240', sinkTemperature=1.0, 
    surface=mdb.models['Model-3'].rootAssembly.surfaces['F-2'])
mdb.models['Model-3'].interactions['FR'].setValues(ambientTemperature=1.0, 
    ambientTemperatureAmp='Fire-240', distributionType=UNIFORM, emissivity=0.8, 
    field='', radiationType=AMBIENT, surface=
    mdb.models['Model-3'].rootAssembly.surfaces['F-2'])
mdb.models['Model-3'].interactions['NF'].setValues(definition=PROPERTY_REF, 
    interactionProperty='NF', sinkAmplitude='Fire-240', sinkTemperature=1.0, 
    surface=mdb.models['Model-3'].rootAssembly.surfaces['NF-2'])

mdb.Model(name='Model-4', objectToCopy=mdb.models['Model-3'])
mdb.models['Model-4'].interactions['F'].setValues(definition=PROPERTY_REF, 
    interactionProperty='F', sinkAmplitude='Fire-240', sinkTemperature=1.0, 
    surface=mdb.models['Model-3'].rootAssembly.surfaces['F-1'])
mdb.models['Model-4'].interactions['FR'].setValues(ambientTemperature=1.0, 
    ambientTemperatureAmp='Fire-240', distributionType=UNIFORM, emissivity=0.8, 
    field='', radiationType=AMBIENT, surface=
    mdb.models['Model-3'].rootAssembly.surfaces['F-1'])
mdb.models['Model-4'].interactions['NF'].setValues(definition=PROPERTY_REF, 
    interactionProperty='NF', sinkAmplitude='Fire-240', sinkTemperature=1.0, 
    surface=mdb.models['Model-3'].rootAssembly.surfaces['NF-1'])

#Job
mdb.Job(atTime=None, contactPrint=OFF, description='', echoPrint=OFF, 
    explicitPrecision=SINGLE, getMemoryFromAnalysis=True, historyPrint=OFF, 
    memory=90, memoryUnits=PERCENTAGE, model='Model-1', modelPrint=OFF, 
    multiprocessingMode=DEFAULT, name='F-240-4s', nodalOutputPrecision=SINGLE, 
    numCpus=8, numDomains=8, numGPUs=0, queue=None, resultsFormat=ODB, scratch=
    '', type=ANALYSIS, userSubroutine='', waitHours=0, waitMinutes=0)

mdb.Job(atTime=None, contactPrint=OFF, description='', echoPrint=OFF, 
    explicitPrecision=SINGLE, getMemoryFromAnalysis=True, historyPrint=OFF, 
    memory=90, memoryUnits=PERCENTAGE, model='Model-2', modelPrint=OFF, 
    multiprocessingMode=DEFAULT, name='F-240-3s', nodalOutputPrecision=SINGLE, 
    numCpus=8, numDomains=8, numGPUs=0, queue=None, resultsFormat=ODB, scratch=
    '', type=ANALYSIS, userSubroutine='', waitHours=0, waitMinutes=0)
	
mdb.Job(atTime=None, contactPrint=OFF, description='', echoPrint=OFF, 
    explicitPrecision=SINGLE, getMemoryFromAnalysis=True, historyPrint=OFF, 
    memory=90, memoryUnits=PERCENTAGE, model='Model-3', modelPrint=OFF, 
    multiprocessingMode=DEFAULT, name='F-240-2s', nodalOutputPrecision=SINGLE, 
    numCpus=8, numDomains=8, numGPUs=0, queue=None, resultsFormat=ODB, scratch=
    '', type=ANALYSIS, userSubroutine='', waitHours=0, waitMinutes=0)
	
mdb.Job(atTime=None, contactPrint=OFF, description='', echoPrint=OFF, 
    explicitPrecision=SINGLE, getMemoryFromAnalysis=True, historyPrint=OFF, 
    memory=90, memoryUnits=PERCENTAGE, model='Model-4', modelPrint=OFF, 
    multiprocessingMode=DEFAULT, name='F-240-1s', nodalOutputPrecision=SINGLE, 
    numCpus=8, numDomains=8, numGPUs=0, queue=None, resultsFormat=ODB, scratch=
    '', type=ANALYSIS, userSubroutine='', waitHours=0, waitMinutes=0)

#mdb.jobs['F-240'].submit(consistencyChecking=OFF)

#Done
\end{lstlisting}

\end{document}